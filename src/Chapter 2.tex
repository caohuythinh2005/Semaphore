\section{Khái niệm cơ bản}

Đèn báo là biến tài nguyên $S$, khởi tạo bằng khả năng phục vụ của tài nguyên nó điều độ. Đèn báo $S$ được khởi tạo như sau:

\begin{lstlisting}
struct Semaphore {
    int value;
    struct process* Ptr;
};
\end{lstlisting}

Chỉ có thể thay đổi giá trị bởi hai thao tác cơ bản là \texttt{wait(S)} và \texttt{signal(S)}.

\subsection*{Thao tác \texttt{wait(S)}}

\begin{lstlisting}
void wait(Semaphore S) {
    S.value--;
    if (S.value < 0) {
        Add process to S.Ptr;
        block();
    }
}
\end{lstlisting}

\subsection*{Thao tác \texttt{signal(S)}}

\begin{lstlisting}
void signal(Semaphore S) {
    S.value++;
    if (S.value <= 0) {
        Pull out process P from S.Ptr;
        wakeup(P);
    }
}
\end{lstlisting}

Ngoài ra, chúng ta còn kết hợp thêm hai thao tác khác là \texttt{block()} và \texttt{wakeup(P)}:

\begin{enumerate}
    \item \texttt{block()}: Ngừng tạm thời tiến trình đang thực hiện.
    \item \texttt{wakeup(P)}: Thực hiện tiếp tiến trình P dừng bởi lệnh \texttt{block()}.
\end{enumerate}

% \subsection{Khái niệm A}
% Giải thích A...

% \subsection{Khái niệm B}
% Giải thích B...